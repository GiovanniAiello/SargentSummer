%%%%%%%%%%%%%%%%%%%%%%%%%%%%%%%%%%%%%%%%%%%%
%  Macro2.tex
%  Exam June 1999
%%%%%%%%%%%%%%%%%%%%%%%%%%%%%%%%%%%%%%%%%%%%

\magnification=1200   
\input psfilars
\input plain
\input vertex
\input ppt.sty

\document
\noindent
{\bf Analytical Question 1} (25p)
\vskip.1cm
\noindent
At the beginning of each period, a worker can choose to work at her last
period's wage or draw a new wage. If she draws a new wage, the old wage is
lost and she can start working at the new wage in the following period.
New wages are i.i.d.\ draws from the c.d.f.\ $F$, where $F(0)=0$, $F(M)=1$
for $M<\infty$. The income of an unemployed worker is $b$, which includes
unemployment insurance and the value of home production. 
A worker seeks to maximize 
$
E_0 \sum_{t=0}^{\infty} (1-\mu)^t \beta^t I_t
$,
where $\mu$ is the probability that a worker dies at the end of a period,
$\beta$ is the subjective discount factor, and $I_t$ is the worker's income
in period $t$, i.e., $I_t$ is equal to the wage $w_t$ when employed and the 
income $b$ when unemployed.

\vskip.2cm

\item{a.} Write the Bellman equation for a worker.
\vskip.1cm

\item{b.} Characterize the agent's decision rule in terms of the model's
parameters. How is the reservation wage affected by $\mu$.
 
\vskip.2cm 
\noindent
The economy is populated with a continuum of workers as described. The exogenous
rate of new workers entering the labor market is equal to $\mu$, i.e., the same
as the death rate. New entrants are unemployed and must draw a new wage.

\vskip.2cm

\item{c.} Find an expression for the economy's unemployment rate in terms
of exogenous parameters and the endogenous reservation wage $\bar w$. Discuss the
determinants of the unemployment rate.

\vskip.2cm 
\noindent
We now change the technology so that 
the economy fluctuates between booms ($B$) and recessions ($R$).
In a boom, all employed workers are paid an extra $z>0$.
That is, the income of a worker with wage $w$ is $I_t=w+z$ in a boom, and
$I_t=w$ in a recession. 
The state of the economy is i.i.d.\ and both outcomes have the same probability
$0.5$ of occurring. 

\vskip.2cm
\item{d. } Write the Bellman equation for a worker. (Hint: A worker's state
vector has now two arguments.)

\vskip.1cm
\item{e. } Characterize the worker's reservation wage policy. How does the
reservation wages in booms and recessions compare to each other? 
(Hint: An already employed worker
might want to draw a new wage when the state of the economy changes.)


\vskip.1cm
\item{f. } Interpret the simulated time series of the model in the figure.
$$
\grafone{pic1.ps,height=2.1in}{}
$$




\end

