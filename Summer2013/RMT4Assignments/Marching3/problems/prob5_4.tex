%!TEX root = ../marching3.tex

\begin{homeworkProblem}[Problem 5.4]

  Modify exercise 5.3 by assuming that the household seeks to maximize

  \begin{align} \label{eq:p4e1}
    - \sum_{t=0}^{\infty} \beta^t \left\{ (s_t - b)^2 + \gamma i_t^2\right\}
  \end{align}

  Here $s_t$ measure consumption services that are produced by durables or habits according to

  \begin{align} \label{eq:p4e2}
    s_t &= \lambda h_t + \pi c_t \\
    h _{t+1} &= \delta h_t + \theta c_t
  \end{align}

  where $h_t$ is the stock of the durable good or habit, ($\lambda, \pi, \delta, \theta$) are parameters, and $h_0$ is an initial condition.

  \begin{enumerate}[a.]
    \item Map this problem into an optimal linear regulator problem.
    \item For the same parameter values as in exercise 5.3 and $(\lambda, \pi, \delta, \theta) = (1, 0.5, 0.95, 1)$ compute the optimal policy for the household.
    \item For the same parameter values as in exercise 5.3 and $(\lambda, \pi, \delta, \theta) = (-1, 1, 0.95, 1)$ compute the optimal policy .
    \item Interpret the parameter settings in part b as capturing a model of durable consumption goods, and the settings in part c as giving a model of habit persistence.
  \end{enumerate}

  \vspace{.2in}

  \problemAnswer{


    \begin{enumerate}[a.]
    \item  I also begin this problem by defining the state and control vectors. The only change is that we now need to keep track of $h_t$ as a state variable. This means that $u$ and $x$ are defined as follows:
      \begin{align} \label{eq:p4e3}
          u_t = \begin{pmatrix} i_t \end{pmatrix}
        \quad
          x _t = \begin{pmatrix}  a_t \\ y_t \\ y_{t-1} \\ h_t \\ 1\end{pmatrix}
        \end{align}

        I now proceed to define the matrices $A, B, R, Q, H$ in a manner similar to how I did it for the previous problem. $A$ and $B$ are easily apparent and I write them here.

        \begin{align} \label{eq:p4e4}
          A =
          \begin{bmatrix}
            1 & 0 & 0 &0 &0 \\
            0 & \rho_1 & \rho_2 & 0 &0 \\
            0 & 1 & 0 &0 & 0 \\
            \theta r & \theta & 0 & \delta & 0 \\
            0 & 0 & 0 & 0 & 1
          \end{bmatrix}
          \quad
          B =
          \begin{bmatrix}
            1\\
            0\\
            0\\
            -\theta \\
            0
          \end{bmatrix}
        \end{align}

        To identify $R$, $Q$, and $H$ I will take a similar approach -- I will make substitutions, expand the result, and do pattern matching to determine the matrices. I will not show all the work again on this problem (it is quite tedious and my approach was clear from the last problem), so I will just give the new values for the matrices.

        \begin{align} \label{eq:p3e5}
          R =
          \begin{bmatrix}
            \pi ^2 r^2 & \pi ^2 r & 0 & \pi  r \lambda  & -b \pi  r \\
             \pi ^2 r & \pi ^2 & 0 & \pi  \lambda  & -b \pi  \\
             0 & 0 & 0 & 0 & 0 \\
             \pi  r \lambda  & \pi  \lambda  & 0 & \lambda ^2 & -b \lambda  \\
             -b \pi  r & -b \pi  & 0 & -b \lambda  & b^2 \\
          \end{bmatrix}
          \quad
          Q = \gamma +\pi ^2
          \quad
          H = \begin{bmatrix} -\pi ^2 r & -\pi ^2 & 0 & -\pi  \lambda  & b \pi  \\ \end{bmatrix}
        \end{align}

        \item I used my \texttt{riccati} equation from above to get the values of $F$ and $P$ that solve this problem. They appear below.

        $$F =
        \begin{bmatrix}
          0.330 & 1.735 & -0.663 & -0.314 & 0.000\\
        \end{bmatrix}
        $$

        $$
        P =
        \begin{bmatrix}
          6.073 & 43.806 & -12.583 & 5.283 & -323.482\\
          43.806 & 319.130 & -91.547 & 38.687 & -2350.346\\
          -12.583 & -91.547 & 26.318 & -10.932 & 669.849\\
          5.283 & 38.687 & -10.932 & 5.494 & -307.692\\
          -323.482 & -2350.346 & 669.849 & -307.692 & 18000.000\\
        \end{bmatrix}
        $$

        \item I repeated the process from part b under different parameters and got the following (note the code for this part and part b are in Listing \ref{code:p54}):

        $$
        F =
        \begin{bmatrix}
        0.179 & 0.688 & -0.425 & -0.170 & -0.000\\
      \end{bmatrix}
        $$

        $$
        P =
        \begin{bmatrix}
          5.051 & 36.406 & -10.429 & 3.945 & 276.113\\
          36.406 & 265.521 & -75.948 & 28.943 & 2006.177\\
          -10.429 & -75.948 & 21.797 & -8.198 & -571.761\\
          3.945 & 28.943 & -8.198 & 5.996 & 307.692\\
          276.113 & 2006.177 & -571.761 & 307.692 & 18000.000\\
        \end{bmatrix}
        $$

        \item For parameterizations in parts b and c, it is instructive to write the laws of motion for $h_{t+1}$ and $s_t$. They appear in the table below.

          \begin{center}
          \begin{tabular}{r| l l}
            & Part b & Part c \\
            \hline
            $h_{t+1}$ & $0.95 h_t + c_t$ & $0.95 h_t + c_t$\\
            $s_t$ & $h_t + 0.05 c_t$ & $c_t - h_t$
          \end{tabular}
          \end{center}

          Notice that the law of motion for $h_{t+1}$ is the same in both cases. The main difference is in how $s_t$ changes from one period to the next. In part b, we have that $s_t = h_t + 0.05c_t$. In this case we see that the marginal effect on the objective function of a change in either $h_t$ or $c_t$ is positive. However, we also see that the effect from a unit increase in $h_t$ is much more influential than the effect of a unit increase in $c_t$. This difference in relative effects results in durable goods not being as meaningful in the period in which they are purchased, thus allowing the parameterization to lead to a model of durable consumption goods.

          In part c, $h_t$ represents the value of today's "habit" The objective function is influenced by $s_t = c_t - h_t$, which is clearly interpreted as the value of today's consumption less today's habit. For this reason, an optimizing agent cannot indefinitely increase the value of the habit each period, because that would drive the objective function further and further negative. This effect makes the parameterization capture a model of habit persistence.

    \end{enumerate}

    \setstretch{0.68}
    \lstinputlisting[language=Python, linerange={38-69}, caption= Solution to 5.3c, label=code:p54]{../RMT4_ch5.py}
    \setstretch{1.5}
    \qed

  }
\end{homeworkProblem}
